\documentclass[../../main.tex]{subfiles}

\begin{document}
\newgeometry {
    a4paper,
    total={170mm,257mm},
    left=30mm,
    top=20mm,
    bottom=30mm,
}

\subsection{Matrix Arithmetic}

\subsubsection{Adding and Subtracting Matrices}
\paragraph{ }
\defn{Matrix Sum} 
$\bA \in \fR^{m \times n}, \bB \in \fR^{m \times n}$, 
\begin{equation*}
    \bA + \bB = \printmatsq{a} + \printmatsq{b} = \printmatsq{(a+b)}.
\end{equation*}

\paragraph{ }
\defn{Scalar Product} 
$\bA \in \fR^{m \times n}, c \in \fR$ 
\begin{equation*}
    c\bA = c \printmatsq{a} = \printmatsq{c a}.
\end{equation*}

\paragraph{ }
\excr{Exercise: }
Define matrix subtraction using Matrix sum and scalar product. 
\begin{equation*}
    \bA - \bB = 
\end{equation*}



\subsubsection{Multiplying Matrices}

\paragraph{ }
\defn{Dot Product}
\index{dot|product}
$\ba, \bb \in \fR^{n}$
\begin{equation}
    \ba  \cdot \bb = \ba^t \bb = \printcolvec{a} \printrowvec{b} = \sum_{i=1}^{n} a_{i} b_{i}.
\end{equation}
\begin{itemize}
    \item Also called inner product.
\end{itemize}

\paragraph{ }
\defn{Matrix Product}
$\bA \in \mathbb{R}^{n \times m}, \bB \in \mathbb{R}^{m \times n}$
\begin{equation*}
    \bA \bB = \printmatrow{a} \cdot \printmatcol{b} = 
    \begin{bmatrix}
        \ba_1 \cdot \bb_1 & \cdots & \ba_1 \cdot \bb_n \\
        \vdots & \ddots & \vdots \\ 
        \ba_m \cdot \bb_1 & \cdots & \ba_m \cdot \bb_n 
    \end{bmatrix}
\end{equation*}

\addNote{
    Matrix and vectors can also be multiplied 
    \begin{equation*}
        \bA \mathbf{x} = \printmatrow{a} \printcolvecsolid{x} = \begin{bmatrix} \ba_1 \cdot \bx \\ \vdots \\ \ba_n \cdot \bx\end{bmatrix}.
    \end{equation*}
}
\excr{Exercise: }{
    Prove that the matrix product between matrices $A \in \fR^{m \times n}$ and $A \in \fR^{l \times k}$ is only possible iff $n = k$.
}


\subsubsection{Matrix Inverse}

\addDef{Matrix Inverse}{
    Given a matrix $\bA \in \fR^{n \times n}$, it's inverse is another matrix $\bA^{-1} \in \fR^{n \times n}$ such that
    \begin{equation*}
        \bA \bA^{-1} = \bA^{-1} \bA = \mathbf{I}.
    \end{equation*}
    \begin{itemize}
        \item 
    \end{itemize}
}

\addDef{Matrix Division} {
    We can define a matrix division using the concept of matrix inverse. 
    \begin{equation*}
        \frac{\bA}{\bB} = \bB^{-1} \bA.
    \end{equation*}
}

\end{document}
